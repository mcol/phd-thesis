%Started on Wednesday, 9th August 2006
% 2007
%Feb: 12
%Mar: 12, 20, 30, 31
%Apr:  2,  3
%Sep: 18, 24

%
% Abstract
%

This research studies two computational techniques that improve
the practical performance of existing implementations 
of interior point methods for linear programming.
Both are based on the concept of symmetric neighbourhood 
as the driving tool for the analysis of the good 
performance of some practical algorithms. 

The symmetric neighbourhood adds explicit upper bounds
on the complementarity pairs, besides the lower bound already
present in the common $\Nhood_{-\infty}$ neighbourhood.
This allows the algorithm to keep under control the spread among 
complementarity pairs and reduce it with the barrier parameter $\mu$.
We show that a long-step feasible algorithm based on 
this neighbourhood is globally convergent and
converges in $\bigO(nL)$ iterations.

The use of the symmetric neighbourhood and the recent
theoretical understanding of the behaviour of Mehrotra's
corrector direction motivate the introduction of a weighting
mechanism that can be applied to any corrector direction,
whether originating from Mehrotra's predictor--corrector algorithm
or as part of the multiple centrality correctors technique.
Such modification in the way a correction is applied
aims to ensure that any computed search direction can positively
contribute to a successful iteration by increasing the overall
stepsize, thus avoiding the case that a corrector is rejected.
The usefulness of the weighting strategy is documented through
complete numerical experiments on various sets of publicly
available test problems.
The implementation within the \HOPDM interior point code
shows remarkable time savings for large-scale linear programming problems.

The second technique develops an efficient way of 
constructing a starting point for structured large-scale 
stochastic linear programs.
We generate a computationally viable warm-start point by solving 
to low accuracy a stochastic problem of much smaller dimension.
The reduced problem is the deterministic equivalent program
corresponding to an event tree composed of a restricted number
of scenarios.
The solution to the reduced problem is then expanded to the
size of the problem instance, and used to initialise the
interior point algorithm.
We present theoretical conditions that the warm-start iterate
has to satisfy in order to be successful.
We implemented this technique in both the \HOPDM and the \OOPS
frameworks, and its performance is verified through 
a series of tests on problem instances coming from various stochastic
programming sources.
