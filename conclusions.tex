%Started on 5 February 2007
%Mar: 31
%Apr:  1

%
% Chapter: Conclusions
%
\label{ch:Conclusions}

The relative youth of interior point methods means that there is still
a lot to learn and try, particularly in practical implementations.
We present here some observations derived from the experience
gathered during this research.

As we have seen in Chapter~\ref{ch:Correctors}, many attempts
have been made to find new and original search directions.
We believe that the direction generated from the Newton system,
possibly complemented by Mehrotra's second-order correction,
are only one of the possibility of exploring the solution space.

From the study of subspace searches explored in
Section~\ref{sec:SubspaceSearches}, it is clear that the more 
directions we consider, the better the final search direction 
we get. Therefore, if we had a cheap way of generating search
directions (rather than from solving a system of linear equations),
then these should be employed.
In this respect, Mehrotra and Li \cite{MehrotraLi} 
mention employing previous search directions alongside the usual ones. 
The use of these incurs an increased memory usage 
in order to store them, but no additional computational cost.
However, it does not seem that they were actually employed in
their implementation.
This opens some questions on what constitutes a valid
previous direction (only affine scaling, the final composite direction
or something else).

The analysis of Jarre and Wechs discussed in Section~\ref{sec:JarreWechs}
makes it very clear that the choice of the target $t$ in
the right-hand side of the Newton system is the driving
tool in finding effective search directions.
In our reimplementation of multiple centrality correctors, we
have pushed the target vector of complementary points further
in the infeasible space with the aim to generate a better
correction to the current iterate.

A big avenue of research, according to the author, is the development
of specialised techniques to exploit the problem structure.

This means that the development and diffusion of structure-exploiting
codes and of structure-aware modelling languages may become a necessary
requirement for a new generation of interior point codes.

In this sense, also theoretical developments in these aspects are 
wanted and necessary.
