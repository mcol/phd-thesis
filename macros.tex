%
% Item point
%
\renewcommand{\labelitemi}{$\circ$}

%
% Framed box
%
\addtolength{\fboxsep}{0.5ex}
\newcommand{\fb}[1]{
 \begin{center}
    \fbox{\parbox{.9\textwidth}{#1}}
 \end{center}
}

%
% Bounded text
%
\newcommand{\bt}[1]{
\begin{center}
\parbox{.9\textwidth}{
\hrulefill 
\vspace{-2ex} 
#1
 \vspace{-3ex}
\hrulefill
} 
\end{center}
\vspace{\parsep}
}

%
% Set of real numbers
%
\newcommand{\R}{\mathbb{R}}

%
% Big-O
%
\newcommand{\bigO}{\mathcal{O}}

%
% Expectation
%
\newcommand{\E}{I\!\!E}

%
% Equations
%
\newcommand{\be}{\begin{equation}}
\newcommand{\ee}{\end{equation}}

%
% Difficult names
%
\newcommand{\yildirim}{Y{\i}ld{\i}r{\i}m\ }

%
% Theorems
%
\newtheorem{theorem}{Theorem}[chapter]
\newtheorem*{theorem*}{Theorem}
\newtheorem{lemma}[theorem]{Lemma}
\newtheorem*{lemma*}{Lemma}
\newtheorem{proposition}[subsection]{Proposition}
\newtheorem{corollary}[subsection]{Corollary}
\newtheorem{claim}[subsection]{Claim}
\newtheorem{condition}[subsection]{Condition}
\newtheorem{conjecture}[subsection]{Conjecture}

%
% Definitions
%
\theoremstyle{definition}
\newtheorem{defn}[subsection]{Definition}
\newtheorem*{defn*}{Definition}
\newtheorem{example}[subsection]{Example}
\newtheorem{examples}[subsection]{Examples}
\newtheorem{note}[subsection]{Note}
\newtheorem{question}[subsection]{Question}
\newtheorem*{question*}{Question}
\newtheorem{remark}{Remark}[chapter]
\newtheorem*{remark*}{Remark}

%
% Multiline comments
%
\newcommand{\ignore}[1]{}
